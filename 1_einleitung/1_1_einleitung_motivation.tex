\section{Einleitung}
\subsection{Motivation und Problemstellung}
Die Energiewende hin zu Klimaneutralität erfordert technologische Innovationen, Investitionen und politische Maßnahmen in allen Sektoren. Insbesondere der Ausbau erneuerbarer Energien, allen voran die Windenergie, spielt dabei in Deutschland eine wichtige Rolle. Ende 2021 waren in Deutschland 28.230 Windkraftanlagen mit einer Gesamtleistung von 56 \gls{gw} installiert, die mit einer Stromerzeugung von 90,3 \gls{twh} pro Jahr knapp 16\% zur gegenwärtigen Stromversorgung beitrugen. Um das Ziel der Klimaneutralität bis 2050 zu erreichen, plant die Bundesregierung einen massiven Ausbau der Windenergie auf 150-200 \gls{gw} installierte Leistung, was einer Vervierfachung der heutigen Kapazität entspricht. % \vglfootcite[seite]{quellenangabe}  

Für den weiteren Ausbau der Windenergie ist die Verfügbarkeit und Transparenz von Informationen über bestehende Anlagen von zentraler Bedeutung. Die Bundesregierung stellt über das Marktstammdatenregister im Rahmen ihrer Open-Data-Initiative umfangreiche Daten zu Windkraftanlagen öffentlich bereit, allerdings sind diese in ihrer rohen, tabellarischen Form für viele Nutzer/-innen nur schwer zugänglich und interpretierbar. Besonders der räumliche Bezug der Anlagen, der für zahlreiche Analysen essentiell ist, lässt sich aus den komplexen Tabellen nur schwer erschließen.

Das zentrale Problem liegt somit in der Diskrepanz zwischen der Verfügbarkeit wertvoller Daten einerseits und deren mangelnder Zugänglichkeit andererseits. Eine intuitive Visualisierung dieser Daten würde nicht nur die Transparenz im Windenergiesektor erhöhen, sondern auch die Grundlage für fundierte Entscheidungen im Kontext der Energiewende schaffen. %\vglfootcite[seite]{quelle}

Die vorliegende Arbeit adressiert diese Problemstellung durch die Konzeption und Entwicklung einer webbasierten Visualisierungsplattform. Mithilfe moderner Webtechnologien wie \gls{nextjs} und \gls{mapbox} sollen die geografischen Daten der Windkraftanlagen auf einer interaktiven Karte dargestellt werden. Die Verwendung von \gls{supabase} als Backend-Lösung ermöglicht dabei eine effiziente Datenverwaltung und -bereitstellung. Ziel ist es, eine benutzerfreundliche Lösung zu schaffen, die die wertvollen Daten des Marktstammdatenregisters einer breiten Öffentlichkeit zugänglich macht.


% Hier weiteren Input einfügen
% \input{xyz.tex}